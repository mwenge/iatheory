\chapter{We Need to Talk About Binary} 
\lstset{style=6502Style}

Here we are, ground zero in every book about assembly programming. Where the reader
finds themselves in an awkdward sit-down discussion about the B word. It's a necessary evil
of course. Nothing will ever make sense without a rudimentary but slowly growing
grasp of what bytes are and why they keep being referred to in a confusing mixture
of letters and numbers such as \icode{\$D012} and \icode{\$F4}.

So what am I supposed to do with you here? Succeed where every other author has
usually failed? I assume that's the expectation. OK here goes.

You already know, OK maybe you already know, that computers, even those in the 
1980s, understand everything in terms of 1s and 0s. All it knows is a world
in which something is on or off, high or low, present or absent. The reason
for this is simply because yes/no and on/off are so fundamental to the reality
of very small things. Once you break anything down far enough you are always
left with a simple question of there being somethere or nothing. The idea of a
\icode{1} or a \icode{0} as the simplest possible building blocks for every
single thing is not so outrageous. You put enough 1s and 0s together in a row
you can build patterns that repeat and create larger patterns and suddenly
you find yourself with a string of 1s and 0s that means something.

The idea of on/off is so simple even a computer can understand it. If you feed
it a signal all it has to do is determine whether it is one or the other and
store the result. Then read the next signal. And before long the computer
has a sequence of 1s and 0s it might be able to do something with.

So when you are programming a computer or in our case trying to understand
how it was programmed to give us Iridis Alpha you have to be able to visualize
how the computer has stored all the ones and zeros it has been given and how
it shuffles them around. An unruly mound of 1s and 0s is not much use when
operating on them intensively. There has to be some benefit to splitting
up the blob in some way that allows some order to be imposed. Some way
of segmenting a string of 1s and 0s that is both useful to us as the
programmers and an efficient way of directing the computer to make effective
use of the amorphous blob fed into it.

Trial and error has eventually arrived at an optimal arrangement, one which
you have definitely heard of: this is the \icode{byte}. The idea of the byte
is simply to divide and conquer. We take any string of 1s and 0s and split
them up into segments of eight. We now consider ourselves to have a string
of bytes and we call each of the individual 1s and 0s \icode{bits}. 

Any scheme that reduces the number of items we have to deal with is a boon,
but is there any rhyme or reason to choosing 8 as our magic number instead
of say 7, or 12? Believe it or not, the number 8 was chosen almost solely
after much experimentation with others because it proved the easiest and
most convenient for people to deal with when understanding how they would
make computers work.

The reason a byte of 8 bits is so convenient to deal with is because of
the number of different values it can represent and the compact notation
this particular number allows. Having a convenient notation system that
allows humans to compose larger values from smaller values is surprisingly
important when it comes down to it. It is humans doing the important part
of the computing: making the big decisions, figuring out where things go
and how they should work. The computer is just a glorified bit shuffler
for which everything is on or off and there is no bigger picture. Humans
need to be able to at least intuit some of this shuffling with a mental
model of what happens when one set of ones and zeros is clashed with another.
Using the magic number of 8 as the denominator for batches of bits enables
this simple mental model.

The first great thing about 8 is that it can be divided into two. That is,
into two segments of 4 bits. The great thing about a sequence of four 1s
and 0s is that it has 16 possible permutations. That's to say it can be
ordered in sixteen unique ways. Yet another way, is to say that a sequence
of 4 bits can store up to sixteen different values.

\begin{figure}[H]
  {
    \setlength{\tabcolsep}{3.0pt}
    \setlength\cmidrulewidth{\heavyrulewidth} % Make cmidrule = 
    \begin{adjustbox}{width=4cm,center}

      \begin{tabular}{rcccccccc}
        \toprule
        Value & Bits & Value & Bits  \\
        \midrule
        0 & \icode{0000} & 8 & \icode{1000}  \\
        1 & \icode{0001} & 9 & \icode{1001}  \\
        2 & \icode{0010} & 10 & \icode{1010}  \\
        3 & \icode{0011} & 11 & \icode{1011}  \\
        4 & \icode{0100} & 12 & \icode{1100}  \\
        5 & \icode{0101} & 13 & \icode{1101}  \\
        6 & \icode{0110} & 14 & \icode{1110}  \\
        7 & \icode{0111} & 15 & \icode{1111}  \\
        \addlinespace
        \bottomrule
      \end{tabular}
    \end{adjustbox}
  }\caption*{The 16 different permutations of four 0s and 1s.}
\end{figure}

You have to agree this is marvellous stuff. But what is so convenient
about a number between 0 and 15? If you're using decimal notation it's
not convenient at all, in some cases you end up having to write two
whole characters rather than one. We could avoid that if we invented
a system that allowed a single character for each of the 16 values.

For example \icode{0-9} for 0-9, and \icode{A-F} for 10-15. Seems a bit
clunky. But it works.

\begin{figure}[H]
  {
    \setlength{\tabcolsep}{3.0pt}
    \setlength\cmidrulewidth{\heavyrulewidth} % Make cmidrule = 
    \begin{adjustbox}{width=4cm,center}

      \begin{tabular}{rcccccccc}
        \toprule
        Value & Bits & Value & Bits  \\
        \midrule
        0 & \icode{0000} & 8 & \icode{1000}  \\
        1 & \icode{0001} & 9 & \icode{1001}  \\
        2 & \icode{0010} & A & \icode{1010}  \\
        3 & \icode{0011} & B & \icode{1011}  \\
        4 & \icode{0100} & C & \icode{1100}  \\
        5 & \icode{0101} & D & \icode{1101}  \\
        6 & \icode{0110} & E & \icode{1110}  \\
        7 & \icode{0111} & F & \icode{1111}  \\
        \addlinespace
        \bottomrule
      \end{tabular}
    \end{adjustbox}
  }\caption*{The 16 different permutations of four 0s and 1s but with letters too.}
\end{figure}

On the other hand, congratulations, we've just invented hexadecimal.

The genius of this system is that it allows us to write every possible
sequence of 8 bits with just two characters. 

\begin{figure}[H]
  {
    \setlength{\tabcolsep}{3.0pt}
    \setlength\cmidrulewidth{\heavyrulewidth} % Make cmidrule = 
    \begin{adjustbox}{width=6cm,center}

      \begin{tabular}{rcccccccc}
        \toprule
        Value & Bits & Value & Bits  \\
        \midrule
        00 & \icode{00000000} & 80 & \icode{10000000}  \\
        01 & \icode{00000001} & 81 & \icode{10000001}  \\
        02 & \icode{00000010} & 82 & \icode{10000010}  \\
        . & . & . & .  \\
        7D & \icode{01111101} & FD & \icode{11111101}  \\
        7E & \icode{01111110} & FE & \icode{11111110}  \\
        7F & \icode{01111111} & FF & \icode{11111111}  \\
        \addlinespace
        \bottomrule
      \end{tabular}
    \end{adjustbox}
  }\caption*{The 256 different permutations of eight 0s and 1s.}
\end{figure}

But that is not its only benefit. When we look at any specific number
using this notation we can almost immediately tell how the bits in it
are set, without needing to memorize all 256 possible permutations.

This is possible because we can start by looking at either character
in the two-character notation and quickly divine the 1s and 0s that
make it up. We can do this because each character represents a value
between 0 and 15 that made is up of some combination of 8, 4, 2, and
1 - each representing a bit from left to right.

\begin{figure}[H]
  {
    \setlength{\tabcolsep}{3.0pt}
    \setlength\cmidrulewidth{\heavyrulewidth} % Make cmidrule = 
    \begin{adjustbox}{width=4cm,center}

      \begin{tabular}{rcccccccc}
        \toprule
        8 & 4 & 2 & 1 & Decimal & Hex  \\
        \midrule
        \icode{0} & \icode{0} & \icode{0} & \icode{0} & 0 & \icode{0}  \\
        \icode{0} & \icode{0} & \icode{0} & \icode{1} & 1 & \icode{1}  \\
        \icode{0} & \icode{0} & \icode{1} & \icode{0} & 2 & \icode{2}  \\
        . & . & . & .  \\
        \icode{1} & \icode{1} & \icode{0} & \icode{1} & 13 & \icode{E}  \\
        \icode{0} & \icode{1} & \icode{1} & \icode{0} & 14 & \icode{D}  \\
        \icode{1} & \icode{1} & \icode{1} & \icode{1} & 15 & \icode{F}  \\
        \addlinespace
        \bottomrule
      \end{tabular}
    \end{adjustbox}
  }\caption*{The 16 different permutations of four 0s and 1s but with letters too.}
\end{figure}

"All" we have to do when confronted with a single hexadecimal digit like \icode{9}
is figure out that its made up of 8 and 1 in order to know that the bits are set
as follows:

\begin{figure}[H]
  {
    \setlength{\tabcolsep}{3.0pt}
    \setlength\cmidrulewidth{\heavyrulewidth} % Make cmidrule = 
    \begin{adjustbox}{width=4cm,center}

      \begin{tabular}{rcccccccc}
        \toprule
        8 & 4 & 2 & 1 & Decimal & Hex  \\
        \midrule
        \icode{1} & \icode{0} & \icode{0} & \icode{1} & 9 & \icode{9}  \\
        \addlinespace
        \bottomrule
      \end{tabular}
    \end{adjustbox}
  }
\end{figure}

Likewise with the slightly more inscrutable \icode{D}. Since \icode{F} is 15,
\icode{D} must be 13. And 13 is made up of 8, 4, and 1:

\begin{figure}[H]
  {
    \setlength{\tabcolsep}{3.0pt}
    \setlength\cmidrulewidth{\heavyrulewidth} % Make cmidrule = 
    \begin{adjustbox}{width=4cm,center}

      \begin{tabular}{rcccccccc}
        \toprule
        8 & 4 & 2 & 1 & Decimal & Hex  \\
        \midrule
        \icode{1} & \icode{1} & \icode{0} & \icode{1} & 13 & \icode{D}  \\
        \addlinespace
        \bottomrule
      \end{tabular}
    \end{adjustbox}
  }
\end{figure}

The ability to do this is especially important when doing the low
level programming that is typical of assembly. We'll see plenty of
examples in future chapters but here is a taste of the kind of operation
we ask a computer to perform by expressing our intentions using 
hexadecimal notation for binary.

Imagine we have a value from some random source stored in a byte. Imagine
we happen to be in need of a random number, for example to select a random
planet to use the next time we run attract mode in Iridis Alpha.

This random number we have stored in a byte sounds like it should be useful to us, but since its
random it can be any value between 0 and 255 and since we only have four
planets to choose from what we need is a number between 0 and 3.

A simple binary operation provides a way of solving this problem and 
our hexadecimal notation gives us a straightforward way of expressing it.

First the binary operation itself. Let's say our random value is a string of bits
like \icode{01011010} which is \icode{5A} in hexadecimal.

\begin{figure}[H]
  {
    \setlength{\tabcolsep}{3.0pt}
    \setlength\cmidrulewidth{\heavyrulewidth} % Make cmidrule = 
    \begin{adjustbox}{width=6cm,center}

      \begin{tabular}{rccccccccc}
        \toprule
        8 & 4 & 2 & 1 & 8 & 4 & 2 & 1 & Decimal & Hex  \\
        \midrule
        \icode{0} & \icode{1} & \icode{0} & \icode{1} & \icode{1} & \icode{0} & \icode{1} & \icode{0} & 90 & \icode{5A}\\
        \addlinespace
      \end{tabular}
    \end{adjustbox}
  }
\end{figure}

Since we only want a value between 0 and 3 we can
extract out just the last two bits to give us that number. In other words, if we 
set all the other bits to zero but that last two we would have a number between 0 and 3:

\begin{figure}[H]
  {
    \setlength{\tabcolsep}{3.0pt}
    \setlength\cmidrulewidth{\heavyrulewidth} % Make cmidrule = 
    \begin{adjustbox}{width=6cm,center}

      \begin{tabular}{rccccccccc}
        \toprule
        8 & 4 & 2 & 1 & 8 & 4 & 2 & 1 & Decimal & Hex  \\
        \midrule
        \icode{0} & \icode{0} & \icode{0} & \icode{0} & \icode{0} & \icode{0} & \bicode{1} & \bicode{0} & 2 & \icode{02}\\
        \addlinespace
      \end{tabular}
    \end{adjustbox}
  }
\end{figure}

We can ask the
computer to do this us by asking it to compare \icode{01011010} with \icode{00000011}
and give use as a result all the bits where both are 1.

\begin{figure}[H]
  {
    \setlength{\tabcolsep}{3.0pt}
    \setlength\cmidrulewidth{\heavyrulewidth} % Make cmidrule = 
    \begin{adjustbox}{width=8cm,center}

      \begin{tabular}{rcccccccc}
        \toprule
        Byte & Bit 7 & Bit 6 & Bit 5 & Bit 4 & Bit 3 & Bit 2 & Bit 1 & Bit 0        \\
        \midrule
        \$5A & 0 & 1 & 0 & 1 & 1 & 0 & \textbf{1} & \textbf{0} \\
        \$03 & 0 & 0 & 0 & 0 & 0 & 0 & \textbf{1} & \textbf{1} \\
        \midrule
        Result: \$02 & 0 & 0 & 0 & 0 & 0 & 0 & \textbf{1} & \textbf{0}\\
        \addlinespace
        \bottomrule
      \end{tabular}
    \end{adjustbox}
  }\caption*{AND'ing \icode{\$5A} and \icode{\$03} gives \icode{\$01} (1). For \icode{AND} to give a 1 both bits must 1 or both must be 0.}
\end{figure}


Notice that in this new bit string we've set the bits that we're interested in
to \icode{1}.  This is called an \icode{AND} operation. We have used
\icode{00000011} to mask out the bits we're not interested in and hence turn a
random number between 0 and 255 into a random number between 0 and 3.

Our choice of 8 as a magic number becomes slightly less magical when we realize
that we never had all possible numbers to choose from in the first place. Sure
we could have chosen 7 or 9 or 13 but how practical would this really be when
the actual value assigned to any individual bit is always going to be a power of
2. Even if we could never precisely intuit the reasons as to why, a magic number
that is also the power of 2, e.g. 8,4,16,64, is always going to make things easier
over the long haul even if we can't articulate them.

So now some jargon we might have heard in the past is starting to make sense. The
Commodore 64 was an '8-bit computer'. What this means is that its basic unit of
currency is the 8-bit byte we've invented here. Everytime it performs an operation
it does so using a single byte as its fundamental building block. This is true 
whether we talking about the \icode{AND} operation or storing a value for use
later on. When we give the C64 a value it's always a single byte. The subsequent
ear of 16-bit computing (e.g. the Amiga) took this a step further by introducing
a two-byte block as its basic unit of currency. 32-bit and 64-bit computing are the
lingua franc of modern processors - no matter how small the value we're using it
will be managed as 4-byte or 8-byte value respectively by 32-bit and 64-bit
computer architectures.

But even the C64 had use for values larger than 255. In fact it could handle staggering
numbers as large as 65,536 using a little sleight of hand to put two bytes side by side
and treat them as a single number.

This was as far as its ken could reach however. And since it could look no further than
65,536 that imposed an upper bound on the number of values it could reach for at any given
time. This finite horizon is known as its address space. When we want to store a value
in the C64's memory we only have 65,536 slots available to specify or select from. The
address we choose must be between \icode{\$0000} (0) and \icode{\$FFFF} (65,536).

\begin{figure}[H]
  {
    \setlength{\tabcolsep}{3.0pt}
    \setlength\cmidrulewidth{\heavyrulewidth} % Make cmidrule = 
    \begin{adjustbox}{width=9cm,center}
      \begin{tikzpicture}
        \draw[step=1.0,gray,thin] (0,0) grid (8,2);
        \fill[gray] (3,1) rectangle ++ (2,1);
        \node[matrix of math nodes,anchor=south west,inner sep=0pt,
              nodes={draw,minimum size=1cm,anchor=center},
              column sep=-\pgflinewidth,row sep=-\pgflinewidth,font=\ttfamily]
              {
                43 & 36 & 34 & 00 & 40 & .. & 41 & 50 & 45 & DE & A2 \\
        \icode{\$0000} & \icode{\$0001} & \icode{\$0002} & \icode{\$0003} & \icode{\$0004}  & .. & \icode{\$FFFC}  & \icode{\$FFFC}  & \icode{\$FFFD}  & \icode{\$FFFE}  & \icode{\$FFFF} \\
						  };
      \end{tikzpicture}
    \end{adjustbox}
  }\caption{The C64's memory visualized as a long strip of 65,536 slots, each holding a byte represented in hexadecimal.}
\end{figure}

This upper limit is what gives the Commodore 64 its name. It has 64 kilobytes of memory.

\begin{figure}[H]
  {
    \setlength{\tabcolsep}{3.0pt}
    \setlength\cmidrulewidth{\heavyrulewidth} % Make cmidrule = 
    \begin{adjustbox}{width=8cm,center}

      \begin{tabular}{ccccccccc}
        \toprule
        Hex & Decimal & Meaning &  & Hex & Decimal & Meaning \\
        \midrule
        \icode{\$0400} & 1024 & 1 Kilobyte &  & \icode{\$0400} & 1024 & 1 Kilobyte & \\\\
        \icode{\$FFFF} & 65,536 & 64 Kilobytes & &  \icode{\$0400} & 1024 & 1 Kilobyte & \\\\
        \bottomrule
      \end{tabular}
    \end{adjustbox}
  }\caption*{Some magic numbers when dealing with bytes.}
\end{figure}

This is all we need to get started. The rest we can pick up along the way. So let's start.










