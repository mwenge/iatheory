\chapter{A Pause Mode for your Pause Mode} 
\lstset{style=6502Style}

Any pause mode must surely be in need of a pause mode. Titled 'DNA' this little entertainment is
a cousin of Minter's previous work on Psychedelia for the C64 and Colorspace for the Atari 800
in 1984 and 1985. It isn't accessed directly from the game but instead is invoked by pressing the
asterisk key while playing 'Made in France'.

Minter first shared it as a tiny 11K demo in a UK Compunet forum in the summer of 
1986. It followed shortly after 'Torus', an oscillator-based demo, shared at the same time and which we
cover elsewhere : both are sprite-based
light synthesisers where, like Psychedelia and Colorspace, the player gets to experiment with different
configurations that control the behaviour of a frantic assembly of brightly colored, pulsating sprites.

DNA has an unapologetically daft premise: two chains of flashing eyeballs cascade down the
screen in an unsettling, blinking helix configuration. Your object as player is to twiddle the available
knobs to see if you can get them to do anything interesting while listening to your favorite music.

For its time DNA's most noteworthy feature was the number of sprites being written to the screen. There are
48 eyeballs displayed at any one time, in addition to a pair of parallax starfields drifting past in the
background. As you may have gathered by now, the C64 can only support 8 sprites in total so this would
have seemed like wizardry to the uninitiated. If you're not dipping into this chapter at random, but have
read any of the previous chapters in this book you may already be able to guess the secret to this 
unsettling feat: raster interrupts.

As elsewhere in Iridis Alpha the trick to filling the screen with sprites is to write a tight piece of code
that can run periodically during a single paint of the screen, adding a layer of sprites to each horizontal
section. We tell the C64 where we want the raster to interrupt its progress and call our code. This code
will then paint as many sprites as possible on the horizontal layer. DNA takes the approach of painting a
pair of eyeballs at 8 pixel vertical intervals, so each horizontal layer is 8 pixels tall. 
These intervals are defined in \icode{dnaSpritesYPositionsArray}:

\begin{lstlisting}
dnaSpritesYPositionsArray       .BYTE $30,$38,$40,$48,$50,$58,$60,$68
                                .BYTE $70,$78,$80,$88,$90,$98,$A0,$A8
                                .BYTE $B0,$B8,$C0,$C8,$D0,$D8,$E0,$E8
                                .BYTE END_SENTINEL
\end{lstlisting}

While the Y co-ordinates of the sprites are set in stone, their X co-ordinates must be calculated on the fly
and indeed the purpose of the twiddling knobs is to control the way these co-ordinates are generated. Initially
the X positions are all set to 192 (\$C0):

\begin{lstlisting}
dnaSpritesXPositionsArray       .BYTE $C0,$C0,$C0,$C0,$C0,$C0,$C0,$C0
                                .BYTE $C0,$C0,$C0,$C0,$C0,$C0,$C0,$C0
                                .BYTE $C0,$C0,$C0,$C0,$C0,$C0,$C0,$C0
                                .BYTE $00,$00,$00,$00,$00,$00,$00,$00
                                .BYTE $00,$00,$00,$00,$00,$00,$00,$00
\end{lstlisting}


\begin{figure}[H]
    \centering
    \foreach \l in {1, ...,24}
    {
      \includegraphics[width=3cm]{dna/dna\l.png}%
    }%
\caption{The first screen paint in DNA. There are 24 raster interrupts allowing us to paint a long chain of sprites.}
\end{figure}

\begin{figure}[H]
  {
    \setlength{\tabcolsep}{1.0pt}
    \setlength\cmidrulewidth{\heavyrulewidth} % Make cmidrule = 
    \begin{adjustbox}{width=14cm,center}
      \begin{tabular}{ccccccc}
        \toprule
        Sprite0 & Sprite1 & Sprite2 & Sprite3 & Sprite4 & Sprite5 & Sprite6 \\
        \midrule
\makecell[l]{
	\begin{subfigure}{0.3\textwidth}
    \def\MULTICOLORONE{gray}
    \def\MULTICOLORTWO{black}
    \def\SPRITECOLOR{yellow}
		\input{sprites/BIG_I}
	\end{subfigure}
} &
\makecell[l]{
	\begin{subfigure}{0.3\textwidth}
    \def\MULTICOLORONE{gray}
    \def\MULTICOLORTWO{black}
    \def\SPRITECOLOR{green}
		\input{sprites/BIG_R}
	\end{subfigure}
} &
\makecell[l]{
	\begin{subfigure}{0.3\textwidth}
    \def\MULTICOLORONE{gray}
    \def\MULTICOLORTWO{black}
    \def\SPRITECOLOR{lightblue}
		\input{sprites/BIG_I}
	\end{subfigure}
} &
\makecell[l]{
	\begin{subfigure}{0.3\textwidth}
    \def\MULTICOLORONE{gray}
    \def\MULTICOLORTWO{black}
    \def\SPRITECOLOR{purple}
		\input{sprites/BIG_D}
	\end{subfigure}
} &
\makecell[l]{
	\begin{subfigure}{0.3\textwidth}
    \def\MULTICOLORONE{gray}
    \def\MULTICOLORTWO{black}
    \def\SPRITECOLOR{blue}
		\input{sprites/BIG_I}
	\end{subfigure}
} &
\makecell[l]{
	\begin{subfigure}{0.3\textwidth}
    \def\MULTICOLORONE{gray}
    \def\MULTICOLORTWO{black}
    \def\SPRITECOLOR{gray}
		\input{sprites/IG_S}
	\end{subfigure}
} &
\makecell[l]{
	\begin{subfigure}{0.3\textwidth}
    \def\MULTICOLORONE{gray}
    \def\MULTICOLORTWO{gray}
    \def\SPRITECOLOR{gray}
		\input{sprites/ALPHA}
	\end{subfigure}
} \\ 
        \midrule
\makecell[l]{
	\begin{subfigure}{0.3\textwidth}
    \def\MULTICOLORONE{gray}
    \def\MULTICOLORTWO{white}
    \def\SPRITECOLOR{red}
		\input{sprites/LAND_GILBY1}
	\end{subfigure}
} & 
\makecell[l]{
	\begin{subfigure}{0.3\textwidth}
    \def\MULTICOLORONE{gray}
    \def\MULTICOLORTWO{white}
    \def\SPRITECOLOR{red}
		\input{sprites/LAND_GILBY2}
	\end{subfigure}
} & 
\makecell[l]{
	\begin{subfigure}{0.3\textwidth}
    \def\MULTICOLORONE{gray}
    \def\MULTICOLORTWO{white}
    \def\SPRITECOLOR{orange}
		\input{sprites/LAND_GILBY3}
	\end{subfigure}
} & 
\makecell[l]{
	\begin{subfigure}{0.3\textwidth}
    \def\MULTICOLORONE{gray}
    \def\MULTICOLORTWO{white}
    \def\SPRITECOLOR{yellow}
		\input{sprites/LAND_GILBY4}
	\end{subfigure}
} & 
\makecell[l]{
	\begin{subfigure}{0.3\textwidth}
    \def\MULTICOLORONE{gray}
    \def\MULTICOLORTWO{white}
    \def\SPRITECOLOR{green}
		\input{sprites/LAND_GILBY5}
	\end{subfigure}
} & 
\makecell[l]{
	\begin{subfigure}{0.3\textwidth}
    \def\MULTICOLORONE{gray}
    \def\MULTICOLORTWO{white}
    \def\SPRITECOLOR{lightblue}
		\input{sprites/LAND_GILBY6}
	\end{subfigure}
} & 
\makecell[l]{
	\begin{subfigure}{0.3\textwidth}
    \def\MULTICOLORONE{gray}
    \def\MULTICOLORTWO{white}
    \def\SPRITECOLOR{purple}
		\input{sprites/LAND_GILBY7}
	\end{subfigure}
} \\ 
        \addlinespace
        \bottomrule
      \end{tabular}
    \end{adjustbox}
  }\caption{The sprites used by the top half of the screen and the bottom half of the screen.}
\end{figure}
