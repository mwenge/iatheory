\chapter{Preface} 
This book describes the inner workings of a relatively obscure video game created
by an eccentric Englishman in 1986 for the Commodore 64.

If you are curious about old computers such as the C64 or the detailed mechanics
of making a glorified digital breadboard produce something on a screen that flashes, bleeps,
and fascinates then this book is hopefully for you. Iridis Alpha was not an enormous success on its release
but it is widely regarded as one of the great achievements on the Commodore 64 hardware. It embodies an 8-bit arcade aesthetic that was
unique to its time, it was slightly mad in its concept and execution, and its emphasis on speed and unforgiving gameplay influenced
generations of game developers in the years that followed.

Since the source code of Iridis Alpha is no longer available, I have had to unpack and reinterpret it
from the game binary Jeff Minter released into the public domain in 2019. I describe this
reverse-engineering process in the opening chapters because I think it is an interesting exercise in and of it self to go from
a binary blob to a set of fully commented source code that provides insight to the inner workings of the game. Hopefully
you will enjoy it too.

The full source code is available in \href{https://github.com/mwenge/iridisalpha}{\textcolor{blue}{its own Github repository}}. 
You should find that it matches exactly the snippets of code provided in the book, though in some cases the extracts in the book have been edited
and reformatted for brevity.

If you are just interested in learning about the mechanics of the game you can flick straight to \hyperref[sec:first16]{\textcolor{blue}{The First 16 Milliseconds}}, 
which begins by covering the first few milliseconds of loading the title screen and all that goes on in there. \hyperref[sec:planets]{\textcolor{blue}{Making Planets for Nigel}},
\hyperref[sec:blast]{\textcolor{blue}{Blasting Fast and Slow}}, and \hyperref[sec:level]{\textcolor{blue}{Enemies and Their Discontents}} cover the main gameplay
and dive into how the game levels are created and how the speed of the game is achieved. 

I dedicate a full chapter \hyperref[sec:first16]{\textcolor{blue}{A Hundred Thousand Billion Theme Tunes}} to the ingenious procedural programming behind Iridis Alpha's theme
tunes and trace its inspiration back to an article 'Musical Fractals' in a 1986 issue of Byte magazine. \hyperref[sec:torusmusic]{\textcolor{blue}{Another 16\textsuperscript{4} Tunes}} looks
at the early expermentiations with this fractal music in the demo 'Torus' Minter released while developing the game. \hyperref[sec:torus]{\textcolor{blue}{An Oscillator in 4 Parts}} delves
into the animation experiment in 'Torus', something that was used in the bonus screen in Iridis Alpha itself.

Iridis Alpha is full of little additional extras. \hyperref[sec:bonus]{\textcolor{blue}{Congoatulations Hotshot!}} unpacks the vertical
scrolling mini-game Minter inserted as a bonus sequence. Even if the game itself is a little half-baked this bonus sequence is full of intrepid effects and some clever techniques
for generating the game maps. \hyperref[sec:mif]{\textcolor{blue}{Made in France}} and \hyperref[sec:dna]{\textcolor{blue}{A Pause Mode for Your Pause Mode}} unpick the pause mode games
bundled with the main game.

As a final coda, \hyperref[sec:bugs]{\textcolor{blue}{Iridis Oops!}} takes a look at some of the bugs that slipped through to the final release. 

If you are reading the PDF version of this book the \hyperref[sec:appendices]{\textcolor{blue}{Appendices}} contain a data dump of sprite sheets, character sets, maps and tables that 
should provide hours of rewarding bedtime reading.

\subsection{Note on the Text}
The version you are reading is little more than a scrapbook still many revisions away from a
finished product. If you find the writing hard going or the attempts to explain things difficult
to follow, by all means \href{https://github.com/mwenge/iatheory/issues}{\textcolor{blue}{leave me a note}} and
I will gratefully accept your complaint. In the meantime, please skip over any blemishes
to the next pretty picture or promising-looking block of text.


- Rob Hogan 2023 \href{https://mastodon.social/@mwenge}{\textcolor{blue}{@mwenge}}
