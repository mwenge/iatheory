\chapter*{About This Book} 
This book describes the inner workings of a relatively obscure video game created
by an eccentric Englishman by the name of Jeff Minter in 1986 for the Commodore 64.

Iridis Alpha was not an enormous success on its release but it is widely regarded as
one of the great achievements on Commodore 64 hardware. It embodies an
8-bit arcade aesthetic that was unique to its time, it was slightly mad in its
concept and execution, and its emphasis on speed and unforgiving gameplay
influenced generations of game developers in the years that followed.

But this book is not just the description of an old computer game. 
It is a tour through computer technologies that could justifiably be considered
antique yet which are still relevant both to the creation of 
computer games and to the programming of computers in general.

Modern computer games can involve hundreds of people in their creation, but in 
1986 the majority of games were conceived and executed by individuals working
entirely on their own like Jeff Minter. The difference is not just one of scale but complexity.
The detail that underlies a computer game from 1986 can fit between two covers
with relative ease because the programmer was working directly with pixels on the
screen (not to mention soundwaves from a speaker) and was constrained by physical
limitations to achieve their goals using as few computer cycles and as little
computer memory and storage as possible. Modern computer games on the other hand are a layer cake
of infinitely dense detail we couldn't possibly hope to capture in their totality
in a single book. With something as small as Iridis Alpha on the other hand,
which is just 54,000 bytes, we can 
understand its operation at the lowest level and set out the challenges and
solutions involved in making a very fast game on slow hardware. The techniques and
tricks we discover are as important today as they were in 1986.

This book is also an introduction 'by example' to the lingua franca of all computers
(even those we use today): assembly language. The flavour of assembly language we learn about here,
6502 Assembly, is no longer in contemporary use but it is one of the
ancestors of the languages used in modern processors. Its relative 
simplicity and clarity makes it the ideal primer for learning the principles
of programming in 'machine code' and understanding how computers work at a fundamental
level. Strip away the abstraction and indirection provided by modern programming
languages such as Python, Java, or even 'C', and you are left with enormous mounds
of apparently indecipherable assembly language instructions. In this book I hope
to show you that these instructions can be deciphered, and were once a viable
way of programming computers.

Finally, this book cannot help being a very specific description of the way
the Commodore 64 itself worked. The Commodore 64 was one of the most successful
home microcomputers ever made. Along with the Apple II and the BBC Micro it was
computing's first step into the home. Although they enabled new uses of computers,
and marked the advent of video games in people's living rooms, these machines were still
simple creatures. And because the Commodore 64 was such a basic computer
in so many ways, what we end up learning about in this book is a recognisable implementation
of a foundational concept in information theory:
\href{https://en.wikipedia.org/wiki/Turing_machine}{\textcolor{blue}{the Turing Machine}}.
64 kilobytes of integrated circuit memory is almost all the C64 has to play with: there is no hard 
disk to speak of. So on our journey we find ourselves with a long Turing-style memory 'tape' on which
our assembly language instructions operate, hopping back and forth along the 
65,536 slots on the tape writing and reading values which the computer hardware
translates into sound and vision. Understanding the code of a video game on an old computer, it
turns out, is a great way to introduce ourselves to the practical details of
computer science's first 'big idea'.

On the other hand, maybe you are just curious. Maybe you just want to learn the detailed mechanics of how to make a
glorified digital breadboard such as the Commodore 64 produce something on a screen\index{screen} that
flashes, bleeps, and fascinates the viewer. If so, then this book was written for you. 

\section*{Overview of the Chapters}
Since the source code of Iridis Alpha is no longer available, I have had to
\href{https://github.com/mwenge/iridisalpha}{\textcolor{blue}{unpack and
reinterpret}} it from the game binary Jeff Minter
\href{https://www.llamasoftarchive.org/oldsite/llamasoft/cbm64/IridisAlpha.zip}{\textcolor{blue}{released
into the public domain in 2019}}. I describe this reverse-engineering process
in the opening chapters because I think it is an interesting exercise in and of
itself to go from a binary blob to a set of fully commented source code that
provides insight to the inner workings of the game.
\hyperref[sec:archaeo]{\textcolor{blue}{A Little Archaeology}} describes how we
extract the game binary from the cassette tape on which it was originally
distributed. \hyperref[sec:disassembly]{\textcolor{blue}{Some Disassembly
Required}} shows you how to go from a very long list of bytes to a full source
code listing. Hopefully you are here because you enjoy this kind of gory detail
too.

If you are just interested in learning about the mechanics of the game itself
you can flick straight to \hyperref[sec:first16]{\textcolor{blue}{The First 16
Milliseconds}}, which begins by covering the loading the title
screen\index{screen} and all that goes on in there.
\hyperref[sec:planets]{\textcolor{blue}{Making Planets for Nigel}},
\hyperref[sec:blast]{\textcolor{blue}{Blasting Fast and Slow}}, and
\hyperref[sec:level]{\textcolor{blue}{Enemies and Their Discontents}} cover the
main gameplay and dive into how the game levels are created and how the speed
of the game is achieved. These chapters go deep into the game's reassembled
code and look closely at how Jeff Minter designed its core engine.

I dedicate a full chapter \hyperref[sec:first16]{\textcolor{blue}{A Hundred
Thousand Billion Theme Tunes}} to the ingenious procedural\index{procedural}
programming behind Iridis Alpha's theme tunes and trace its inspiration back to
an article 'Musical Fractals' in a 1986 issue of Byte magazine.
\hyperref[sec:torusmusic]{\textcolor{blue}{Another 16\textsuperscript{4}
Tunes}} looks at the early expermentiations with this fractal music in the demo
program 'Torus' that Minter released while developing Iridis Alpha. It also
delves into the animation\index{animation} experiment it contains, something
that was used in the bonus screen\index{screen} in Iridis Alpha itself.

Iridis Alpha is full of little additional extras.
\hyperref[sec:bonus]{\textcolor{blue}{Congoatulations Hotshot!}} unpacks the
vertical scrolling mini-game Minter inserted as a bonus
sequence\index{sequence}. Even if the game itself is a little half-baked this
bonus sequence\index{sequence} is full of intrepid effects and some clever
techniques for generating the game maps.
\hyperref[sec:mif]{\textcolor{blue}{Made in France}} and
\hyperref[sec:dna]{\textcolor{blue}{A Pause Mode for Your Pause Mode}} unpick
the pause mode games bundled with the main game.

As a final coda, \hyperref[sec:bugs]{\textcolor{blue}{Iridis Oops!}} takes a
look at some of the bugs that slipped through to the final release. 

If you are reading the PDF version of this book the
\hyperref[sec:appendices]{\textcolor{blue}{Appendices}} contain a data dump of
sprite sheets, character\index{character} sets, maps and tables that should
provide hours of rewarding bedtime reading, or more likely, deep undisturbed
sleep.

\section*{Note on the Text} If you find the writing hard going or the attempts
to explain things difficult to follow, by all means
\href{https://github.com/mwenge/iatheory/issues}{\textcolor{blue}{leave me a
note}} and I will gratefully accept your complaint. In the meantime, please
skip over any blemishes to the next pretty picture or promising-looking block
of text.

The full source code is available in
\href{https://github.com/mwenge/iridisalpha}{\textcolor{blue}{its own Github
repository}}.  You should find that it matches exactly the snippets of code
provided in the book, though in some cases the extracts in the book have been
edited and reformatted for brevity.

\href{https://mastodon.social/@mwenge}{\textcolor{blue}{Rob Hogan}}\\
Dublin, 2023 - 2024 \\

\vspace*{\fill}
\begin{figure}[H]
    \centering
      \includegraphics[width=5cm]{src/cover/title_page.png}%
\end{figure}
\vspace*{\fill}
\thispagestyle{empty}%
\clearpage

