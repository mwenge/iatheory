\chapter{Another 16\textsuperscript{4} Tunes} 
\label{sec:torusmusic}

It's possible to dig into the making of the title music and how Jeff Minter
arrived the music configuration he did thanks to a number of tiny demo programs that
survive from the period when he was developing Iridis Alpha.

Jeff distributed these on Compunet in the summer of 1986.

It turns out he was inspired by an article in 'Byte' magazine from June 1986 that
described how to make 'Fractal Music'. This article outline a version of the
algorithm that Jeff ultimately adopted. The 'self-similarity' we encountered
in the way the Iridis Alpha theme tunes are constructed, a four-note structure\index{structure}
repeated across different time intervals on each of the three voices, finds its
roots in this article.


\begin{figure}[H]
{
  \begin{adjustbox}{width=11cm,center}
  \includegraphics[width=11cm]{torus/fractal.jpg}%
    \end{adjustbox}
}\caption[]{}
\end{figure}


\section{Taurus:Torus}
\begin{figure}[H]
{
  \begin{adjustbox}{width=11cm,center}
  \includegraphics[width=11cm]{torus/torus.png}%
    \end{adjustbox}
}\caption[]{}
\end{figure}

This first demo, released in July 1986(?), has a version of Iridis' music-generating
algorithm that is nearly fully formed. However, the music it produces is quite
different. In fact, it is nearer to a tool for listening to and selecting music than
anything else.

The four seed values in \icode{titleMusicNoteArray\index{titleMusicNoteArray}} that are used to seed all subequently
generated tunes (\icode{00 07 0C 07} in Iridis Alpha) can be selected and changed by the
user. They're called 'Oscillators' and each can be any value between 0 and 16, i.e. any of
\icode{0 1 2 3 4 5 6 7 8 9 A B C D E F}.


\lstset{style=6502Style}
\lstinputlisting[caption=The music routine\index{routine} in Torus:Taurus side-by-side with Iridis Alpha.,basicstyle=\tiny]{torus/sidebyside.asm}

When it runs the demo cycles through procedural\index{procedural} configurations of \icode{titleMusicNoteArray\index{titleMusicNoteArray}} of 64 notes each.
In other words, exactly the kind of fractal structure\index{structure} we observed in Iridis Alpha proper. The examples below
give a flavour of the music it generats:

\begin{figure}[H]
{
  \begin{adjustbox}{width=14cm,center}
    \includegraphics[width=14cm]{torus/title_no_1_page_1001.png}}%
  \end{adjustbox}
}\caption[]{Bars 2 and 4 are always repeated}
\end{figure}


\begin{figure}[H]
{
  \begin{adjustbox}{width=14cm,center}
    \includegraphics[width=14cm]{torus/title_no_2_page_1001.png}}%
  \end{adjustbox}
}\caption[]{Bars 2 and 4 are always repeated}
\end{figure}

Not all of the tunes are 64-note based. It does generate some that are truncated. 

\begin{figure}[H]
  {
    \setlength{\tabcolsep}{3.0pt}
    \setlength\cmidrulewidth{\heavyrulewidth} % Make cmidrule = 
	\centering
	\def\MULTICOLORONE{green}
	\def\MULTICOLORTWO{red}
	\def\SPRITECOLOR{blue}
		\input{sprites/BULLHEAD}
  }\caption[position=top]{The 'Torus' sprite.}
\end{figure}


\section{Taurus/Torus Two}
\begin{figure}[H]
{
  \begin{adjustbox}{width=11cm,center}
  \includegraphics[width=11cm]{torus/torus2.png}%
    \end{adjustbox}
}\caption[]{}
\end{figure}

\lstset{style=6502Style}
\lstinputlisting[caption=The music routine\index{routine} in Taurus:Torus II side-by-side with Iridis Alpha.,basicstyle=\tiny]{torus/sidebyside2.asm}


\begin{figure}[H]
{
  \begin{adjustbox}{width=14cm,center}
    \includegraphics[width=14cm]{torus/torus2_title_no_1_page_1001.png}%
  \end{adjustbox}
}\caption[]{}
\end{figure}

\begin{figure}[H]
{
  \begin{adjustbox}{width=14cm,center}
    \includegraphics[width=14cm]{music/title_no_1_page_1001.png}%
  \end{adjustbox}
}\caption[]{}
\end{figure}

